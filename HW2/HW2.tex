\documentclass[letterpaper, 11pt]{article}
%!TEX TS-program = pdflatex
%Completely insane addition to force textmate to use pdflatex

\special{papersize=8.5in,11in}
\setlength{\pdfpageheight}{\paperheight}
\setlength{\pdfpagewidth}{\paperwidth}
\setlength{\parindent}{0in}
\setlength{\parskip}{0.06in}

\marginparwidth 0pt
\oddsidemargin  0pt
\evensidemargin  0pt
\marginparsep 0pt

\topmargin   0pt

\textwidth   6.5in
\textheight  8.5 in
\usepackage{amssymb, amsmath, graphicx}
\usepackage{epstopdf}
\usepackage{subfigure}
\usepackage{cleveref}
\usepackage{subfigure}
\usepackage{booktabs}
\usepackage{listings}
%\usepackage{algorithm}
%\usepackage[noend]{algorithmic}
%\usepackage{ifthen}
%\usepackage{xspace}

%\usepackage{fullpage}
\DeclareMathOperator*{\argmin}{arg\,min}
\newcommand{\lz}[0]{Lov\'{a}sz }
\newcommand{\lecture}[5]{
   \renewcommand{\thepage}{\arabic{page}}
   \noindent
   \begin{center}
   \framebox{
      \vbox{
    \hbox to 5.78in { {\bf EE381V: Advanced Coding Theory} \hfill #2 }
       \vspace{4mm}
       \hbox to 5.78in { {\Large \hfill #5  \hfill} }
       \vspace{2mm}
       \hbox to 5.78in { {\it Lecturer: #3 \hfill Scribe: #4} }
      }
   }
   \end{center}
   \vspace*{4mm}
}

\newcommand{\homework}[5]{
   \renewcommand{\thepage}{\arabic{page}}
   \noindent
   \begin{center}
   \framebox{
      \vbox{
    \hbox to 5.78in { {\bf EE381V: Machine Learning, Homework #1} \hfill #2 }
       \vspace{4mm}

            \hbox to 5.78in { {\it \hfill #3} }

      }
   }
   \end{center}
   \vspace*{4mm}
}

\newenvironment{problem}{\begin{prob}
                         \rm}{\end{prob}}
\newcommand{\makeline}{\noindent \rule{\textwidth}{0.05in} \newline}
\newcommand{\maketline}{\noindent \rule{\textwidth}{0.025in} \newline}

\newtheorem{theorem}{Theorem}
\newtheorem{definition}[theorem]{Definition}
\newtheorem{remark}[theorem]{Remark}
\newtheorem{lemma}[theorem]{Lemma}
\newtheorem{corollary}[theorem]{Corollary}
\newtheorem{proposition}[theorem]{Proposition}
\newtheorem{claim}[theorem]{Claim}
\newtheorem{observation}[theorem]{Observation}
\newtheorem{fact}[theorem]{Fact}
\newtheorem{assumption}[theorem]{Assumption}
\newtheorem{prob}{Q.}

\def\QED{{\phantom{x}} \hfill $\rule{1.3ex}{1.3ex}$}

\def\proof{\rm \trivlist \item[\hskip \labelsep{\bf Proof. }]}
\def\endproof{\QED \endtrivlist}


\newcommand{\OPT}{\ensuremath{\mathrm{OPT}}\xspace}
\def\OR{orthonormal representation }

\begin{document}

%%%%%%%%%%%%%%%%%%%%%
% Create  header.
\homework
{2}					% Lecture number
{\today}		% Date of lecture
{Jiaxiao Zheng}		% Name of scriber
\section{Computational Problems}

\section{Written Problems}
\begin{problem}
\begin{enumerate}
\item In fig. 3.10, $f(p) = 1-(1-p^4)^4$. Then $f^\prime(p) = 16(1-p^4)^3p^3$.
Set $g(p) = \log(f^\prime(p)) = \log 16 +3\log(1-p^4) +3\log p$. Then maximizing $g(p)$ we can get $p = 0.2^{\frac14} = 0.6687$. The maximal $f^\prime(p) = 2.45$.

In fig. 3.11, $f(p) = (1-(1-p)^4)^4$. Then $f^\prime(p) = 16(1-(1-p)^4)^3(1-p)^3$.
Similarly, set $g(p) = \log(f^\prime(p)) = \log 16 +3\log(1-(1-p)^4) + 3\log(1-p)$. Maximizing $g(p)$ we have $p = 0.3313$. The maximal is again $f^\prime(p) = 2.45$.

\item In a more general settings, if we first apply $r$-way AND construction and then $b$-way OR: $f(p) = 1-(1-p^r)^b$.
Then similarly $\log(f^\prime(p)) = \log br + (b-1)\log(1-p^r) + (r-1)\log p$, with maximizer at $p = (\frac{r-1}{r(b-1) + r - 1})^{\frac{1}{r}}$, and maximal slope $f^\prime(p) = br(1-\frac{r-1}{rb-1})^{b-1}(\frac{r-1}{rb-1})^{\frac{r-1}{r}}$

If we first apply $b$-way OR and then $r$-way AND: $f(p) = (1-(1-p)^b)^r$. $g(p) = \log(f^\prime(p)) = \log br + (r-1)\log(1-(1-p)^b) + (b-1)\log(1-p)$. Maximizer is $p = 1-(\frac{b-1}{b(r-1) + (b-1)})^{\frac1b}$, and maximal slope $f^\prime(p) = br(1-\frac{b-1}{br-1})^{r-1}(\frac{b-1}{br - 1})^{\frac{b-1}{b}}$


\end{enumerate}
\end{problem}

\begin{problem}
If $P$ is $k$-dimensional, $\Lambda = \textrm{diag}([\textrm{$k$ 1's}, 0,0,\dots,0])$.

Since $UPU^T = \Lambda$, $P = U^T\Lambda U$, then $U^T\Lambda ^2 U v = U^T\Lambda Uv$. If that holds true the only possibility is that the diagonal of $\Lambda$ contains only 0 and 1. Given the dimensionality of $P$ and $\Lambda$ are both $k$, it is easy to see we have $k$ 1's on the diagonal and other elements are 0.
\end{problem}

\begin{problem}
\begin{enumerate}
\item If $x\in span\{v_1\}$, $x = \beta v_1$ where $\beta \in \mathbb{R}$. Then the original optimization changes to:
$$
\min_\beta (a-\beta v_1)^T(a-\beta v_1)
$$
equivalent to
$$
\min_\beta f(\beta) =  -2\beta a^Tv_1 + \beta^2 v_1^Tv_1
$$
It is easy to see the optimizer is $\beta^* = \frac{a^Tv_1}{v_1^Tv_1} = a^Tv_1$ because $v_1$ is normalized. Then the optimizer of the original problem is $x = <a,v_1>v_1^T$

\item The original optimization problem can be written as:
\begin{eqnarray*}
\min_v && \|A-Avv^T\|^2_F \\
&& = \sum_{i=1}^n\|a_i - a_ivv^T\|^2_2 \\
&& = \sum_{i=1}^n(a_i - a_ivv^T)(a_i-a_ivv^T)^T \\
&& = \sum_{i=1}^n(a_ia_i^T + a_ivv^Tvv^Ta_i^T - 2a_ivv^Ta_i^T)
\end{eqnarray*}
where the first and second terms are irrelevant to $v$ as long as $v$ is normalized. Therefore it is equivalent to 
\begin{eqnarray*}
\max && \sum_{i = 1}^n a_ivv^Ta_i^T \\
&& = \sum_{i = 1}^n v^Ta_i^Ta_iv \\
&& = v^TA^TAv
\end{eqnarray*}
\end{enumerate}
\end{problem}



\end{document} 